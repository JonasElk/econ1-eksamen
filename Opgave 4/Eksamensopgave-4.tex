% Options for packages loaded elsewhere
\PassOptionsToPackage{unicode}{hyperref}
\PassOptionsToPackage{hyphens}{url}
%
\documentclass[
]{article}
\usepackage{amsmath,amssymb}
\usepackage{iftex}
\ifPDFTeX
  \usepackage[T1]{fontenc}
  \usepackage[utf8]{inputenc}
  \usepackage{textcomp} % provide euro and other symbols
\else % if luatex or xetex
  \usepackage{unicode-math} % this also loads fontspec
  \defaultfontfeatures{Scale=MatchLowercase}
  \defaultfontfeatures[\rmfamily]{Ligatures=TeX,Scale=1}
\fi
\usepackage{lmodern}
\ifPDFTeX\else
  % xetex/luatex font selection
\fi
% Use upquote if available, for straight quotes in verbatim environments
\IfFileExists{upquote.sty}{\usepackage{upquote}}{}
\IfFileExists{microtype.sty}{% use microtype if available
  \usepackage[]{microtype}
  \UseMicrotypeSet[protrusion]{basicmath} % disable protrusion for tt fonts
}{}
\makeatletter
\@ifundefined{KOMAClassName}{% if non-KOMA class
  \IfFileExists{parskip.sty}{%
    \usepackage{parskip}
  }{% else
    \setlength{\parindent}{0pt}
    \setlength{\parskip}{6pt plus 2pt minus 1pt}}
}{% if KOMA class
  \KOMAoptions{parskip=half}}
\makeatother
\usepackage{xcolor}
\usepackage[margin=1in]{geometry}
\usepackage{color}
\usepackage{fancyvrb}
\newcommand{\VerbBar}{|}
\newcommand{\VERB}{\Verb[commandchars=\\\{\}]}
\DefineVerbatimEnvironment{Highlighting}{Verbatim}{commandchars=\\\{\}}
% Add ',fontsize=\small' for more characters per line
\usepackage{framed}
\definecolor{shadecolor}{RGB}{248,248,248}
\newenvironment{Shaded}{\begin{snugshade}}{\end{snugshade}}
\newcommand{\AlertTok}[1]{\textcolor[rgb]{0.94,0.16,0.16}{#1}}
\newcommand{\AnnotationTok}[1]{\textcolor[rgb]{0.56,0.35,0.01}{\textbf{\textit{#1}}}}
\newcommand{\AttributeTok}[1]{\textcolor[rgb]{0.13,0.29,0.53}{#1}}
\newcommand{\BaseNTok}[1]{\textcolor[rgb]{0.00,0.00,0.81}{#1}}
\newcommand{\BuiltInTok}[1]{#1}
\newcommand{\CharTok}[1]{\textcolor[rgb]{0.31,0.60,0.02}{#1}}
\newcommand{\CommentTok}[1]{\textcolor[rgb]{0.56,0.35,0.01}{\textit{#1}}}
\newcommand{\CommentVarTok}[1]{\textcolor[rgb]{0.56,0.35,0.01}{\textbf{\textit{#1}}}}
\newcommand{\ConstantTok}[1]{\textcolor[rgb]{0.56,0.35,0.01}{#1}}
\newcommand{\ControlFlowTok}[1]{\textcolor[rgb]{0.13,0.29,0.53}{\textbf{#1}}}
\newcommand{\DataTypeTok}[1]{\textcolor[rgb]{0.13,0.29,0.53}{#1}}
\newcommand{\DecValTok}[1]{\textcolor[rgb]{0.00,0.00,0.81}{#1}}
\newcommand{\DocumentationTok}[1]{\textcolor[rgb]{0.56,0.35,0.01}{\textbf{\textit{#1}}}}
\newcommand{\ErrorTok}[1]{\textcolor[rgb]{0.64,0.00,0.00}{\textbf{#1}}}
\newcommand{\ExtensionTok}[1]{#1}
\newcommand{\FloatTok}[1]{\textcolor[rgb]{0.00,0.00,0.81}{#1}}
\newcommand{\FunctionTok}[1]{\textcolor[rgb]{0.13,0.29,0.53}{\textbf{#1}}}
\newcommand{\ImportTok}[1]{#1}
\newcommand{\InformationTok}[1]{\textcolor[rgb]{0.56,0.35,0.01}{\textbf{\textit{#1}}}}
\newcommand{\KeywordTok}[1]{\textcolor[rgb]{0.13,0.29,0.53}{\textbf{#1}}}
\newcommand{\NormalTok}[1]{#1}
\newcommand{\OperatorTok}[1]{\textcolor[rgb]{0.81,0.36,0.00}{\textbf{#1}}}
\newcommand{\OtherTok}[1]{\textcolor[rgb]{0.56,0.35,0.01}{#1}}
\newcommand{\PreprocessorTok}[1]{\textcolor[rgb]{0.56,0.35,0.01}{\textit{#1}}}
\newcommand{\RegionMarkerTok}[1]{#1}
\newcommand{\SpecialCharTok}[1]{\textcolor[rgb]{0.81,0.36,0.00}{\textbf{#1}}}
\newcommand{\SpecialStringTok}[1]{\textcolor[rgb]{0.31,0.60,0.02}{#1}}
\newcommand{\StringTok}[1]{\textcolor[rgb]{0.31,0.60,0.02}{#1}}
\newcommand{\VariableTok}[1]{\textcolor[rgb]{0.00,0.00,0.00}{#1}}
\newcommand{\VerbatimStringTok}[1]{\textcolor[rgb]{0.31,0.60,0.02}{#1}}
\newcommand{\WarningTok}[1]{\textcolor[rgb]{0.56,0.35,0.01}{\textbf{\textit{#1}}}}
\usepackage{graphicx}
\makeatletter
\def\maxwidth{\ifdim\Gin@nat@width>\linewidth\linewidth\else\Gin@nat@width\fi}
\def\maxheight{\ifdim\Gin@nat@height>\textheight\textheight\else\Gin@nat@height\fi}
\makeatother
% Scale images if necessary, so that they will not overflow the page
% margins by default, and it is still possible to overwrite the defaults
% using explicit options in \includegraphics[width, height, ...]{}
\setkeys{Gin}{width=\maxwidth,height=\maxheight,keepaspectratio}
% Set default figure placement to htbp
\makeatletter
\def\fps@figure{htbp}
\makeatother
\setlength{\emergencystretch}{3em} % prevent overfull lines
\providecommand{\tightlist}{%
  \setlength{\itemsep}{0pt}\setlength{\parskip}{0pt}}
\setcounter{secnumdepth}{-\maxdimen} % remove section numbering
\usepackage{setspace}\onehalfspacing
\ifLuaTeX
  \usepackage{selnolig}  % disable illegal ligatures
\fi
\IfFileExists{bookmark.sty}{\usepackage{bookmark}}{\usepackage{hyperref}}
\IfFileExists{xurl.sty}{\usepackage{xurl}}{} % add URL line breaks if available
\urlstyle{same}
\hypersetup{
  hidelinks,
  pdfcreator={LaTeX via pandoc}}

\author{}
\date{\vspace{-2.5em}}

\begin{document}

\hypertarget{eksamenssuxe6t-4}{%
\section{Eksamenssæt 4}\label{eksamenssuxe6t-4}}

\hypertarget{opgave-1---opstil-en-lineuxe6r-regressionsmodel-for-hvor-du-bruger-de-beskrevne-forklarende-variable.}{%
\subsection{\texorpdfstring{Opgave 1 - Opstil en lineær regressionsmodel
for \textit{participation} hvor du bruger de beskrevne forklarende
variable.}{Opgave 1 - Opstil en lineær regressionsmodel for  hvor du bruger de beskrevne forklarende variable.}}\label{opgave-1---opstil-en-lineuxe6r-regressionsmodel-for-hvor-du-bruger-de-beskrevne-forklarende-variable.}}

\hypertarget{a---estimer-modellen-vha.-ols-og-kommenter-puxe5-resultaterne.}{%
\subsubsection{(a) - Estimer modellen vha. OLS og kommenter på
resultaterne.}\label{a---estimer-modellen-vha.-ols-og-kommenter-puxe5-resultaterne.}}

MANGLER FORKLARING AF LPM + SVAGHEDER U ER PER DEFINITION
HETEROSKEDASTIC - DER BRUGES ROBUST SE

\begin{Shaded}
\begin{Highlighting}[]
\NormalTok{model\_ols }\OtherTok{=} \FunctionTok{lm}\NormalTok{(participation }\SpecialCharTok{\textasciitilde{}}\NormalTok{ income }\SpecialCharTok{+}\NormalTok{ age }\SpecialCharTok{+}\NormalTok{ agesq }\SpecialCharTok{+}\NormalTok{ educ }\SpecialCharTok{+}\NormalTok{ youngkids }\SpecialCharTok{+}\NormalTok{ oldkids }\SpecialCharTok{+}\NormalTok{ foreign, }\AttributeTok{data =}\NormalTok{ data)}
\NormalTok{robust\_ols }\OtherTok{=} \FunctionTok{coeftest}\NormalTok{(model\_ols, }\AttributeTok{vcov =} \FunctionTok{vcovHC}\NormalTok{(model\_ols, }\AttributeTok{type =} \StringTok{"HC0"}\NormalTok{))}

\FunctionTok{screenreg}\NormalTok{(}\FunctionTok{list}\NormalTok{(}\AttributeTok{OLS =}\NormalTok{ model\_ols, }\AttributeTok{OLS\_robust\_se =}\NormalTok{ robust\_ols), }\AttributeTok{digits =} \DecValTok{4}\NormalTok{)}
\end{Highlighting}
\end{Shaded}

\begin{verbatim}
## 
## ========================================
##              OLS           OLS_robust_se
## ----------------------------------------
## (Intercept)   -0.3686      -0.3686      
##               (0.2530)     (0.2358)     
## income        -0.0035 ***  -0.0035 ***  
##               (0.0007)     (0.0006)     
## age            0.0634 ***   0.0634 ***  
##               (0.0129)     (0.0119)     
## agesq         -0.0009 ***  -0.0009 ***  
##               (0.0002)     (0.0001)     
## educ           0.0068       0.0068      
##               (0.0060)     (0.0059)     
## youngkids     -0.2390 ***  -0.2390 ***  
##               (0.0314)     (0.0302)     
## oldkids       -0.0475 **   -0.0475 **   
##               (0.0172)     (0.0175)     
## foreign        0.2572 ***   0.2572 ***  
##               (0.0401)     (0.0401)     
## ----------------------------------------
## R^2            0.1901                   
## Adj. R^2       0.1836                   
## Num. obs.    872                        
## ========================================
## *** p < 0.001; ** p < 0.01; * p < 0.05
\end{verbatim}

\begin{Shaded}
\begin{Highlighting}[]
\CommentTok{\#summary(model\_ols)}
\end{Highlighting}
\end{Shaded}

Alle signifikante på 0,1\% pånær oldkids på 1\% og educ er ikke
signifikant. INGEN FORSKEL PGA ROBUST SE

\hypertarget{b---test-om-den-partielle-effekt-af-uddannelse-er-forskellig-fra-nul.}{%
\subsubsection{(b) - Test om den partielle effekt af uddannelse er
forskellig fra
nul.}\label{b---test-om-den-partielle-effekt-af-uddannelse-er-forskellig-fra-nul.}}

For at teste hvorvidt den partielle effekt af en variabel er forskellig
fra nul bruges en t-test. Hvorvidt nulhypotesen afvises afhænger af den
beregnede t-score og dertilhørende p-værdi \[H_0: \beta_4 = 0\]
\[H_1: \beta_4 \neq 0\] T-scoren beregnes ud fra den estimerede
\(\beta\) samt den tilhørende standardafvigelse. Dette kan gøres, da
nulhypotesen er, at den faktiske værdi er nul, hvorfor dette led ikke
indgår i formlen. \[t = \frac{\hat{\beta_j}}{se(\hat{\beta_j})}\] I
nedenstående vil de robuste standardafvigelser blive benyttet til
udregningen af t-scoren

\begin{verbatim}
## Kritisk værdi ved 5% =  1.9626913
\end{verbatim}

\begin{verbatim}
## Kritisk værdi vec 1% =  2.5814857
\end{verbatim}

\begin{Shaded}
\begin{Highlighting}[]
\NormalTok{t }\OtherTok{=} \FloatTok{0.0068}\SpecialCharTok{/}\FloatTok{0.0059}
\end{Highlighting}
\end{Shaded}

\begin{verbatim}
## t-score =  1.1525424
\end{verbatim}

\hypertarget{c---test-om-den-partielle-effekt-af-alder-er-forskellig-fra-nul.}{%
\subsubsection{(c) - Test om den partielle effekt af alder er forskellig
fra
nul.}\label{c---test-om-den-partielle-effekt-af-alder-er-forskellig-fra-nul.}}

T-test?

\hypertarget{opgave-2---opstil-buxe5de-en-logit--og-en-probit-model-for-par-t-i-cipat-ion-hvor-du-bruger-de-beskrevne-forklarende-variable.}{%
\subsection{Opgave 2 - Opstil både en logit- og en probit-model for par
t i cipat ion hvor du bruger de beskrevne forklarende
variable.}\label{opgave-2---opstil-buxe5de-en-logit--og-en-probit-model-for-par-t-i-cipat-ion-hvor-du-bruger-de-beskrevne-forklarende-variable.}}

\hypertarget{a---estimer-modellerne.}{%
\subsubsection{(a) - Estimer
modellerne.}\label{a---estimer-modellerne.}}

\begin{Shaded}
\begin{Highlighting}[]
\FunctionTok{screenreg}\NormalTok{(}\FunctionTok{list}\NormalTok{(}\StringTok{"LPM OLS"} \OtherTok{=}\NormalTok{ model\_ols, }\AttributeTok{Logit =}\NormalTok{ logit, }\AttributeTok{Probit =}\NormalTok{ probit))}
\end{Highlighting}
\end{Shaded}

\begin{verbatim}
## 
## ====================================================
##                 LPM OLS     Logit        Probit     
## ----------------------------------------------------
## (Intercept)      -0.37        -4.39 ***    -2.67 ***
##                  (0.25)       (1.30)       (0.78)   
## income           -0.00 ***    -0.02 ***    -0.01 ***
##                  (0.00)       (0.00)       (0.00)   
## age               0.06 ***     0.33 ***     0.20 ***
##                  (0.01)       (0.07)       (0.04)   
## agesq            -0.00 ***    -0.00 ***    -0.00 ***
##                  (0.00)       (0.00)       (0.00)   
## educ              0.01         0.04         0.02    
##                  (0.01)       (0.03)       (0.02)   
## youngkids        -0.24 ***    -1.18 ***    -0.71 ***
##                  (0.03)       (0.17)       (0.10)   
## oldkids          -0.05 **     -0.24 **     -0.14 ** 
##                  (0.02)       (0.08)       (0.05)   
## foreign           0.26 ***     1.19 ***     0.73 ***
##                  (0.04)       (0.20)       (0.12)   
## ----------------------------------------------------
## R^2               0.19                              
## Adj. R^2          0.18                              
## Num. obs.       872          872          872       
## AIC                         1032.15      1031.65    
## BIC                         1070.32      1069.82    
## Log Likelihood              -508.08      -507.83    
## Deviance                    1016.15      1015.65    
## ====================================================
## *** p < 0.001; ** p < 0.01; * p < 0.05
\end{verbatim}

ESTIMATER FOR LOGIT OG PROBIT KAN IKKE FORTOLKES SOM DE ER

\hypertarget{b---test-om-den-partielle-effekt-af-uddannelse-er-forskellig-fra-nul.-1}{%
\subsubsection{(b) - Test om den partielle effekt af uddannelse er
forskellig fra
nul.}\label{b---test-om-den-partielle-effekt-af-uddannelse-er-forskellig-fra-nul.-1}}

\hypertarget{c---test-om-den-partielle-effekt-af-alder-er-forskellig-fra-nul-vha.-et-likelihoodratio-test.}{%
\subsubsection{(c) - Test om den partielle effekt af alder er forskellig
fra nul vha. et
likelihoodratio-test.}\label{c---test-om-den-partielle-effekt-af-alder-er-forskellig-fra-nul-vha.-et-likelihoodratio-test.}}

\hypertarget{opgave-3---vi-vil-gerne-sammenligne-den-partielle-effekt-af-puxe5-tvuxe6rs-af-modellerne.-beregn-average-partial-effect-ape-og-kommenter-puxe5-resultaterne.}{%
\subsection{\texorpdfstring{Opgave 3 - Vi vil gerne sammenligne den
partielle effekt af \textit{income} på tværs af modellerne. Beregn
average partial effect (APE) og kommenter på
resultaterne.}{Opgave 3 - Vi vil gerne sammenligne den partielle effekt af  på tværs af modellerne. Beregn average partial effect (APE) og kommenter på resultaterne.}}\label{opgave-3---vi-vil-gerne-sammenligne-den-partielle-effekt-af-puxe5-tvuxe6rs-af-modellerne.-beregn-average-partial-effect-ape-og-kommenter-puxe5-resultaterne.}}

BRUGER ROBUST SE

\begin{Shaded}
\begin{Highlighting}[]
\NormalTok{ape\_logit }\OtherTok{=} \FunctionTok{logitmfx}\NormalTok{(logit, }\AttributeTok{data =}\NormalTok{ data, }\AttributeTok{atmean=}\NormalTok{F, }\AttributeTok{robust =}\NormalTok{ T)}

\FunctionTok{screenreg}\NormalTok{(}\FunctionTok{list}\NormalTok{(}\AttributeTok{ape\_logit =}\NormalTok{ ape\_logit), }\AttributeTok{digits =} \DecValTok{4}\NormalTok{)}
\end{Highlighting}
\end{Shaded}

\begin{verbatim}
## 
## =============================
##                 ape_logit    
## -----------------------------
## income            -0.0046 ***
##                   (0.0010)   
## age                0.0657 ***
##                   (0.0139)   
## agesq             -0.0009 ***
##                   (0.0002)   
## educ               0.0077    
##                   (0.0060)   
## youngkids         -0.2350 ***
##                   (0.0403)   
## oldkids           -0.0470 ** 
##                   (0.0176)   
## foreign            0.2466 ***
##                   (0.0409)   
## -----------------------------
## Num. obs.        872         
## Log Likelihood  -508.0766    
## Deviance        1016.1533    
## AIC             1032.1533    
## BIC             1070.3196    
## =============================
## *** p < 0.001; ** p < 0.01; * p < 0.05
\end{verbatim}

\begin{Shaded}
\begin{Highlighting}[]
\NormalTok{ape\_logit}
\end{Highlighting}
\end{Shaded}

\begin{verbatim}
## Call:
## logitmfx(formula = logit, data = data, atmean = F, robust = T)
## 
## Marginal Effects:
##                  dF/dx    Std. Err.        z           P>|z|    
## income    -0.004610926  0.001012942 -4.55201 0.0000053135052 ***
## age        0.065744335  0.013887098  4.73420 0.0000021991793 ***
## agesq     -0.000929881  0.000175837 -5.28832 0.0000001234475 ***
## educ       0.007705869  0.006036510  1.27654       0.2017634    
## youngkids -0.235006974  0.040327957 -5.82740 0.0000000056299 ***
## oldkids   -0.046973245  0.017550880 -2.67640       0.0074417 ** 
## foreign    0.246583549  0.040923752  6.02544 0.0000000016865 ***
## ---
## Signif. codes:  0 '***' 0.001 '**' 0.01 '*' 0.05 '.' 0.1 ' ' 1
## 
## dF/dx is for discrete change for the following variables:
## 
## [1] "foreign"
\end{verbatim}

\hypertarget{opgave-4---vi-vil-gerne-sammenligne-den-partielle-effekt-af-puxe5-tvuxe6rs-af-modellerne.-beregn-ape-og-kommenter-puxe5-resultaterne.}{%
\subsection{\texorpdfstring{Opgave 4 - Vi vil gerne sammenligne den
partielle effekt af \textit{foreign} på tværs af modellerne. Beregn APE
og kommenter på
resultaterne.}{Opgave 4 - Vi vil gerne sammenligne den partielle effekt af  på tværs af modellerne. Beregn APE og kommenter på resultaterne.}}\label{opgave-4---vi-vil-gerne-sammenligne-den-partielle-effekt-af-puxe5-tvuxe6rs-af-modellerne.-beregn-ape-og-kommenter-puxe5-resultaterne.}}

\hypertarget{opgave-5---hvorfor-er-ape-at-foretruxe6kke-frem-for-partial-effect-at-the-average-pea}{%
\subsection{Opgave 5 - Hvorfor er APE at foretrække frem for partial
effect at the average
(PEA)?}\label{opgave-5---hvorfor-er-ape-at-foretruxe6kke-frem-for-partial-effect-at-the-average-pea}}

\hypertarget{opgave-6---sammenlign-modellernes-evne-til-at-pruxe6diktere-ved-at-beregne-percent-correctly-predicted-for-hver-model.}{%
\subsection{Opgave 6 - Sammenlign modellernes evne til at prædiktere ved
at beregne percent correctly predicted for hver
model.}\label{opgave-6---sammenlign-modellernes-evne-til-at-pruxe6diktere-ved-at-beregne-percent-correctly-predicted-for-hver-model.}}

\end{document}
