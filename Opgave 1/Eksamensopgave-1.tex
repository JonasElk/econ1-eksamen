% Options for packages loaded elsewhere
\PassOptionsToPackage{unicode}{hyperref}
\PassOptionsToPackage{hyphens}{url}
%
\documentclass[
]{article}
\usepackage{amsmath,amssymb}
\usepackage{iftex}
\ifPDFTeX
  \usepackage[T1]{fontenc}
  \usepackage[utf8]{inputenc}
  \usepackage{textcomp} % provide euro and other symbols
\else % if luatex or xetex
  \usepackage{unicode-math} % this also loads fontspec
  \defaultfontfeatures{Scale=MatchLowercase}
  \defaultfontfeatures[\rmfamily]{Ligatures=TeX,Scale=1}
\fi
\usepackage{lmodern}
\ifPDFTeX\else
  % xetex/luatex font selection
\fi
% Use upquote if available, for straight quotes in verbatim environments
\IfFileExists{upquote.sty}{\usepackage{upquote}}{}
\IfFileExists{microtype.sty}{% use microtype if available
  \usepackage[]{microtype}
  \UseMicrotypeSet[protrusion]{basicmath} % disable protrusion for tt fonts
}{}
\makeatletter
\@ifundefined{KOMAClassName}{% if non-KOMA class
  \IfFileExists{parskip.sty}{%
    \usepackage{parskip}
  }{% else
    \setlength{\parindent}{0pt}
    \setlength{\parskip}{6pt plus 2pt minus 1pt}}
}{% if KOMA class
  \KOMAoptions{parskip=half}}
\makeatother
\usepackage{xcolor}
\usepackage[margin=1in]{geometry}
\usepackage{graphicx}
\makeatletter
\def\maxwidth{\ifdim\Gin@nat@width>\linewidth\linewidth\else\Gin@nat@width\fi}
\def\maxheight{\ifdim\Gin@nat@height>\textheight\textheight\else\Gin@nat@height\fi}
\makeatother
% Scale images if necessary, so that they will not overflow the page
% margins by default, and it is still possible to overwrite the defaults
% using explicit options in \includegraphics[width, height, ...]{}
\setkeys{Gin}{width=\maxwidth,height=\maxheight,keepaspectratio}
% Set default figure placement to htbp
\makeatletter
\def\fps@figure{htbp}
\makeatother
\setlength{\emergencystretch}{3em} % prevent overfull lines
\providecommand{\tightlist}{%
  \setlength{\itemsep}{0pt}\setlength{\parskip}{0pt}}
\setcounter{secnumdepth}{-\maxdimen} % remove section numbering
\usepackage{setspace}\onehalfspacing
\ifLuaTeX
  \usepackage{selnolig}  % disable illegal ligatures
\fi
\IfFileExists{bookmark.sty}{\usepackage{bookmark}}{\usepackage{hyperref}}
\IfFileExists{xurl.sty}{\usepackage{xurl}}{} % add URL line breaks if available
\urlstyle{same}
\hypersetup{
  hidelinks,
  pdfcreator={LaTeX via pandoc}}

\author{}
\date{\vspace{-2.5em}}

\begin{document}

\hypertarget{eksamenssuxe6t-1}{%
\section{Eksamenssæt 1}\label{eksamenssuxe6t-1}}

\hypertarget{opgave-1---estimer-modellen-vha.-ols.-kommenter-puxe5-outputtet-og-fortolk-resultaterne}{%
\subsection{Opgave 1 - Estimer modellen vha. OLS. Kommenter på outputtet
og fortolk
resultaterne}\label{opgave-1---estimer-modellen-vha.-ols.-kommenter-puxe5-outputtet-og-fortolk-resultaterne}}

\hypertarget{opgave-2---udfuxf8r-grafisk-modelkontrol}{%
\subsection{Opgave 2 - Udfør grafisk
modelkontrol}\label{opgave-2---udfuxf8r-grafisk-modelkontrol}}

\hypertarget{opgave-3---test-for-heteroskedasticitet-vha.-breusch-pagan-testet-og-specialudgaven-af-white-testet}{%
\subsection{Opgave 3 - Test for heteroskedasticitet vha.
Breusch-Pagan-testet og specialudgaven af
White-testet}\label{opgave-3---test-for-heteroskedasticitet-vha.-breusch-pagan-testet-og-specialudgaven-af-white-testet}}

\hypertarget{opgave-4---beregn-robuste-standardfejl-for-modellen-og-sammenlign-med-resultaterne-i-spuxf8rgsmuxe5l-1}{%
\subsection{Opgave 4 - Beregn robuste standardfejl for modellen og
sammenlign med resultaterne i spørgsmål
1}\label{opgave-4---beregn-robuste-standardfejl-for-modellen-og-sammenlign-med-resultaterne-i-spuxf8rgsmuxe5l-1}}

\hypertarget{opgave-5---test-hypotesen-h0-beta_2-1-mod-alternativet-h1-beta_2-neq-1}{%
\subsection{\texorpdfstring{Opgave 5 - Test hypotesen H0:
\(\beta_2 = 1\) mod alternativet H1:
\(\beta_2 \neq 1\)}{Opgave 5 - Test hypotesen H0: \textbackslash beta\_2 = 1 mod alternativet H1: \textbackslash beta\_2 \textbackslash neq 1}}\label{opgave-5---test-hypotesen-h0-beta_2-1-mod-alternativet-h1-beta_2-neq-1}}

\hypertarget{opgave-6---test-hypotesen-h0-beta_3-beta_4-0}{%
\subsection{\texorpdfstring{Opgave 6 - Test hypotesen H0:
\(\beta_3 = \beta_4 = 0\)}{Opgave 6 - Test hypotesen H0: \textbackslash beta\_3 = \textbackslash beta\_4 = 0}}\label{opgave-6---test-hypotesen-h0-beta_3-beta_4-0}}

\hypertarget{opgave-7---estimer-modellen-vha.-fgls-og-kommenter-puxe5-resultaterne}{%
\subsection{Opgave 7 - Estimer modellen vha. FGLS og kommenter på
resultaterne}\label{opgave-7---estimer-modellen-vha.-fgls-og-kommenter-puxe5-resultaterne}}

\hypertarget{opgave-8---har-fgls-estimationen-taget-huxf8jde-for-al-heteroskedasticiteten}{%
\subsection{Opgave 8 - Har FGLS estimationen taget højde for al
heteroskedasticiteten?}\label{opgave-8---har-fgls-estimationen-taget-huxf8jde-for-al-heteroskedasticiteten}}

\end{document}
