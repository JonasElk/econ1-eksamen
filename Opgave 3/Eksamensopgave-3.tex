% Options for packages loaded elsewhere
\PassOptionsToPackage{unicode}{hyperref}
\PassOptionsToPackage{hyphens}{url}
%
\documentclass[
]{article}
\usepackage{amsmath,amssymb}
\usepackage{iftex}
\ifPDFTeX
  \usepackage[T1]{fontenc}
  \usepackage[utf8]{inputenc}
  \usepackage{textcomp} % provide euro and other symbols
\else % if luatex or xetex
  \usepackage{unicode-math} % this also loads fontspec
  \defaultfontfeatures{Scale=MatchLowercase}
  \defaultfontfeatures[\rmfamily]{Ligatures=TeX,Scale=1}
\fi
\usepackage{lmodern}
\ifPDFTeX\else
  % xetex/luatex font selection
\fi
% Use upquote if available, for straight quotes in verbatim environments
\IfFileExists{upquote.sty}{\usepackage{upquote}}{}
\IfFileExists{microtype.sty}{% use microtype if available
  \usepackage[]{microtype}
  \UseMicrotypeSet[protrusion]{basicmath} % disable protrusion for tt fonts
}{}
\makeatletter
\@ifundefined{KOMAClassName}{% if non-KOMA class
  \IfFileExists{parskip.sty}{%
    \usepackage{parskip}
  }{% else
    \setlength{\parindent}{0pt}
    \setlength{\parskip}{6pt plus 2pt minus 1pt}}
}{% if KOMA class
  \KOMAoptions{parskip=half}}
\makeatother
\usepackage{xcolor}
\usepackage[margin=1in]{geometry}
\usepackage{color}
\usepackage{fancyvrb}
\newcommand{\VerbBar}{|}
\newcommand{\VERB}{\Verb[commandchars=\\\{\}]}
\DefineVerbatimEnvironment{Highlighting}{Verbatim}{commandchars=\\\{\}}
% Add ',fontsize=\small' for more characters per line
\usepackage{framed}
\definecolor{shadecolor}{RGB}{248,248,248}
\newenvironment{Shaded}{\begin{snugshade}}{\end{snugshade}}
\newcommand{\AlertTok}[1]{\textcolor[rgb]{0.94,0.16,0.16}{#1}}
\newcommand{\AnnotationTok}[1]{\textcolor[rgb]{0.56,0.35,0.01}{\textbf{\textit{#1}}}}
\newcommand{\AttributeTok}[1]{\textcolor[rgb]{0.13,0.29,0.53}{#1}}
\newcommand{\BaseNTok}[1]{\textcolor[rgb]{0.00,0.00,0.81}{#1}}
\newcommand{\BuiltInTok}[1]{#1}
\newcommand{\CharTok}[1]{\textcolor[rgb]{0.31,0.60,0.02}{#1}}
\newcommand{\CommentTok}[1]{\textcolor[rgb]{0.56,0.35,0.01}{\textit{#1}}}
\newcommand{\CommentVarTok}[1]{\textcolor[rgb]{0.56,0.35,0.01}{\textbf{\textit{#1}}}}
\newcommand{\ConstantTok}[1]{\textcolor[rgb]{0.56,0.35,0.01}{#1}}
\newcommand{\ControlFlowTok}[1]{\textcolor[rgb]{0.13,0.29,0.53}{\textbf{#1}}}
\newcommand{\DataTypeTok}[1]{\textcolor[rgb]{0.13,0.29,0.53}{#1}}
\newcommand{\DecValTok}[1]{\textcolor[rgb]{0.00,0.00,0.81}{#1}}
\newcommand{\DocumentationTok}[1]{\textcolor[rgb]{0.56,0.35,0.01}{\textbf{\textit{#1}}}}
\newcommand{\ErrorTok}[1]{\textcolor[rgb]{0.64,0.00,0.00}{\textbf{#1}}}
\newcommand{\ExtensionTok}[1]{#1}
\newcommand{\FloatTok}[1]{\textcolor[rgb]{0.00,0.00,0.81}{#1}}
\newcommand{\FunctionTok}[1]{\textcolor[rgb]{0.13,0.29,0.53}{\textbf{#1}}}
\newcommand{\ImportTok}[1]{#1}
\newcommand{\InformationTok}[1]{\textcolor[rgb]{0.56,0.35,0.01}{\textbf{\textit{#1}}}}
\newcommand{\KeywordTok}[1]{\textcolor[rgb]{0.13,0.29,0.53}{\textbf{#1}}}
\newcommand{\NormalTok}[1]{#1}
\newcommand{\OperatorTok}[1]{\textcolor[rgb]{0.81,0.36,0.00}{\textbf{#1}}}
\newcommand{\OtherTok}[1]{\textcolor[rgb]{0.56,0.35,0.01}{#1}}
\newcommand{\PreprocessorTok}[1]{\textcolor[rgb]{0.56,0.35,0.01}{\textit{#1}}}
\newcommand{\RegionMarkerTok}[1]{#1}
\newcommand{\SpecialCharTok}[1]{\textcolor[rgb]{0.81,0.36,0.00}{\textbf{#1}}}
\newcommand{\SpecialStringTok}[1]{\textcolor[rgb]{0.31,0.60,0.02}{#1}}
\newcommand{\StringTok}[1]{\textcolor[rgb]{0.31,0.60,0.02}{#1}}
\newcommand{\VariableTok}[1]{\textcolor[rgb]{0.00,0.00,0.00}{#1}}
\newcommand{\VerbatimStringTok}[1]{\textcolor[rgb]{0.31,0.60,0.02}{#1}}
\newcommand{\WarningTok}[1]{\textcolor[rgb]{0.56,0.35,0.01}{\textbf{\textit{#1}}}}
\usepackage{graphicx}
\makeatletter
\def\maxwidth{\ifdim\Gin@nat@width>\linewidth\linewidth\else\Gin@nat@width\fi}
\def\maxheight{\ifdim\Gin@nat@height>\textheight\textheight\else\Gin@nat@height\fi}
\makeatother
% Scale images if necessary, so that they will not overflow the page
% margins by default, and it is still possible to overwrite the defaults
% using explicit options in \includegraphics[width, height, ...]{}
\setkeys{Gin}{width=\maxwidth,height=\maxheight,keepaspectratio}
% Set default figure placement to htbp
\makeatletter
\def\fps@figure{htbp}
\makeatother
\setlength{\emergencystretch}{3em} % prevent overfull lines
\providecommand{\tightlist}{%
  \setlength{\itemsep}{0pt}\setlength{\parskip}{0pt}}
\setcounter{secnumdepth}{-\maxdimen} % remove section numbering
\usepackage{setspace}\onehalfspacing
\ifLuaTeX
  \usepackage{selnolig}  % disable illegal ligatures
\fi
\IfFileExists{bookmark.sty}{\usepackage{bookmark}}{\usepackage{hyperref}}
\IfFileExists{xurl.sty}{\usepackage{xurl}}{} % add URL line breaks if available
\urlstyle{same}
\hypersetup{
  hidelinks,
  pdfcreator={LaTeX via pandoc}}

\author{}
\date{\vspace{-2.5em}}

\begin{document}

\hypertarget{eksamenssuxe6t-3-instrumentvariable}{%
\section{Eksamenssæt 3:
Instrumentvariable}\label{eksamenssuxe6t-3-instrumentvariable}}

\hypertarget{opgave-1---estimer-modellen-vha.-ols-og-kommenter-puxe5-resultaterne}{%
\subsection{Opgave 1 - Estimer modellen vha. OLS og kommenter på
resultaterne}\label{opgave-1---estimer-modellen-vha.-ols-og-kommenter-puxe5-resultaterne}}

\begin{Shaded}
\begin{Highlighting}[]
\NormalTok{model }\OtherTok{=} \FunctionTok{lm}\NormalTok{(learnings }\SpecialCharTok{\textasciitilde{}}\NormalTok{ educ }\SpecialCharTok{+}\NormalTok{ exp }\SpecialCharTok{+}\NormalTok{ male }\SpecialCharTok{+}\NormalTok{ ethblack }\SpecialCharTok{+}\NormalTok{ ethhisp, data)}
\FunctionTok{summary}\NormalTok{(model)}
\end{Highlighting}
\end{Shaded}

\begin{verbatim}
## 
## Call:
## lm(formula = learnings ~ educ + exp + male + ethblack + ethhisp, 
##     data = data)
## 
## Residuals:
##       Min        1Q    Median        3Q       Max 
## -2.075849 -0.280064 -0.001448  0.307748  1.984409 
## 
## Coefficients:
##               Estimate Std. Error t value              Pr(>|t|)    
## (Intercept)  0.3962265  0.1735078  2.2836              0.022801 *  
## educ         0.1242201  0.0094515 13.1429 < 0.00000000000000022 ***
## exp          0.0338820  0.0050456  6.7152      0.00000000004986 ***
## male         0.2934491  0.0458032  6.4067      0.00000000033631 ***
## ethblack    -0.1956696  0.0712545 -2.7461              0.006243 ** 
## ethhisp     -0.0974063  0.1003417 -0.9707              0.332132    
## ---
## Signif. codes:  0 '***' 0.001 '**' 0.01 '*' 0.05 '.' 0.1 ' ' 1
## 
## Residual standard error: 0.51025 on 514 degrees of freedom
## Multiple R-squared:  0.35393,    Adjusted R-squared:  0.34764 
## F-statistic: 56.316 on 5 and 514 DF,  p-value: < 0.000000000000000222
\end{verbatim}

hispanics er insignifikant med p-værdi på 33,2\% Resten er på 1\% eller
lavere Relativ lav \(R^2\) F-test med meget lav p-værdi - Variable er
``jointly significant''

\hypertarget{opgave-2---hvorfor-kunne-vi-vuxe6re-bekymrede-for-at-uddannelse-er-endogen}{%
\subsection{Opgave 2 - Hvorfor kunne vi være bekymrede for at uddannelse
er
endogen?}\label{opgave-2---hvorfor-kunne-vi-vuxe6re-bekymrede-for-at-uddannelse-er-endogen}}

Uddannelse vil være endogen hvis den er korreleret med en udeladt
variabel som derfor skaber en bias. Denne udeladte variabel kunne
eksempelvis være ``ability'', som påvirker uddannelsesniveauet positivt
og dermed skaber en bias.

\hypertarget{opgave-3---er-siblings-meduc-og-feduc-brugbare-som-instrumenter}{%
\subsection{Opgave 3 - Er siblings, meduc og feduc brugbare som
instrumenter?}\label{opgave-3---er-siblings-meduc-og-feduc-brugbare-som-instrumenter}}

Hvis de nævnte variable er korreleret med uddannelse, mens de ikke er
korreleret med den udeladte variabel, som i dette tilfælde er
``ability'', vil de være egnet som instrumenter. Det kan formentlig
antages, at forældre uddannelse eller antal søskende uddannelse ikke har
indflydelse på ``ability'', hvorfor denne betingelse til
instrumentvariablen er opfyldt. Samtidig er forældres uddannelsesniveau
formentlig delvist korreleret med den pågældendes uddannelse, mens antal
søskende ikke i samme grad antages at være korreleret med
uddannelsesniveauet. Derfor vil forældres uddannelse formentlig være
bedre instrumenter end antal søskende.

\hypertarget{opgave-4---test-om-uddannelse-er-endogen}{%
\subsection{Opgave 4 - Test om uddannelse er
endogen}\label{opgave-4---test-om-uddannelse-er-endogen}}

Testen for endogenitet laves vha. den reduceret ligning, som er
variablen mistænkt for endogenitetsproblemer regresseret på de øvrige
uafhængige variable og instrumentvariablene. Heri er variablen eksogen,
altså ukorreleret med det oprindelige fejlled (\(u\)), hvis og kun hvis
fejlleddet fra den reducerede ligning (\(v\)) er ukorreleret med
(\(u\)). Fejlleddet fra den reducerede ligning (\(v\)) er dog ikke
observeret, hvorfor residualet bruges som proxy. Derfor inkluderes
(\(v\)) i den oprindelige regresion, hvorefter en t-test bruges til
teste hvorvidt den tilhørende estimator \(\delta\) er signifikant. Hvis
det findes, at \(\delta\) ikke kan siges at være lig 0 er variablen
endogen. Modsat vil variablen antages at være eksogen hvis nulhypotesen
\(H_0: \gamma = 0\) ikke kan afvises.

\begin{Shaded}
\begin{Highlighting}[]
\NormalTok{red\_model }\OtherTok{=} \FunctionTok{lm}\NormalTok{(educ }\SpecialCharTok{\textasciitilde{}}\NormalTok{ exp }\SpecialCharTok{+}\NormalTok{ male }\SpecialCharTok{+}\NormalTok{ ethblack }\SpecialCharTok{+}\NormalTok{ ethhisp }\SpecialCharTok{+}\NormalTok{ siblings }\SpecialCharTok{+}\NormalTok{ meduc }\SpecialCharTok{+}\NormalTok{ feduc, data)}
\NormalTok{v }\OtherTok{=} \FunctionTok{resid}\NormalTok{(red\_model)}

\NormalTok{endo\_model }\OtherTok{=} \FunctionTok{lm}\NormalTok{(learnings }\SpecialCharTok{\textasciitilde{}}\NormalTok{ educ }\SpecialCharTok{+}\NormalTok{ exp }\SpecialCharTok{+}\NormalTok{ male }\SpecialCharTok{+}\NormalTok{ ethblack }\SpecialCharTok{+}\NormalTok{ ethhisp }\SpecialCharTok{+}\NormalTok{ v, data)}
\FunctionTok{summary}\NormalTok{(endo\_model)}
\end{Highlighting}
\end{Shaded}

\begin{verbatim}
## 
## Call:
## lm(formula = learnings ~ educ + exp + male + ethblack + ethhisp + 
##     v, data = data)
## 
## Residuals:
##       Min        1Q    Median        3Q       Max 
## -2.121193 -0.279066 -0.003089  0.298275  2.074787 
## 
## Coefficients:
##              Estimate Std. Error t value           Pr(>|t|)    
## (Intercept) -0.064700   0.339003 -0.1909            0.84871    
## educ         0.153036   0.020516  7.4593 0.0000000000003742 ***
## exp          0.037628   0.005567  6.7591 0.0000000000378421 ***
## male         0.290479   0.045775  6.3458 0.0000000004868537 ***
## ethblack    -0.157544   0.075122 -2.0972            0.03647 *  
## ethhisp     -0.069476   0.101739 -0.6829            0.49499    
## v           -0.036550   0.023106 -1.5818            0.11430    
## ---
## Signif. codes:  0 '***' 0.001 '**' 0.01 '*' 0.05 '.' 0.1 ' ' 1
## 
## Residual standard error: 0.50951 on 513 degrees of freedom
## Multiple R-squared:  0.35706,    Adjusted R-squared:  0.34955 
## F-statistic: 47.484 on 6 and 513 DF,  p-value: < 0.000000000000000222
\end{verbatim}

\hypertarget{opgave-5---estimer-modellen-vha.-2sls-hvor-du-guxf8r-brug-af-de-tre-beskrevne-instrumenter.-sammenlign-med-resultaterne-i-spuxf8rgsmuxe5l-1.}{%
\subsection{Opgave 5 - Estimer modellen vha. 2SLS hvor du gør brug af de
tre beskrevne instrumenter. Sammenlign med resultaterne i spørgsmål
1.}\label{opgave-5---estimer-modellen-vha.-2sls-hvor-du-guxf8r-brug-af-de-tre-beskrevne-instrumenter.-sammenlign-med-resultaterne-i-spuxf8rgsmuxe5l-1.}}

\begin{Shaded}
\begin{Highlighting}[]
\NormalTok{educ\_fitted }\OtherTok{=} \FunctionTok{fitted}\NormalTok{(red\_model)}

\FunctionTok{linearHypothesis}\NormalTok{(red\_model, }\FunctionTok{c}\NormalTok{(}\StringTok{"meduc=0"}\NormalTok{, }\StringTok{"feduc=0"}\NormalTok{, }\StringTok{"siblings=0"}\NormalTok{)) }\CommentTok{\#Test at IVs er signifikante}
\end{Highlighting}
\end{Shaded}

\begin{verbatim}
## Linear hypothesis test
## 
## Hypothesis:
## meduc = 0
## feduc = 0
## siblings = 0
## 
## Model 1: restricted model
## Model 2: educ ~ exp + male + ethblack + ethhisp + siblings + meduc + feduc
## 
##   Res.Df     RSS Df Sum of Sq       F                 Pr(>F)    
## 1    515 2914.55                                                
## 2    512 2297.79  3   616.752 45.8087 < 0.000000000000000222 ***
## ---
## Signif. codes:  0 '***' 0.001 '**' 0.01 '*' 0.05 '.' 0.1 ' ' 1
\end{verbatim}

\begin{Shaded}
\begin{Highlighting}[]
\NormalTok{sls }\OtherTok{=} \FunctionTok{lm}\NormalTok{(learnings }\SpecialCharTok{\textasciitilde{}}\NormalTok{ educ\_fitted }\SpecialCharTok{+}\NormalTok{ exp }\SpecialCharTok{+}\NormalTok{ male }\SpecialCharTok{+}\NormalTok{ ethblack }\SpecialCharTok{+}\NormalTok{ ethhisp, data)}
\CommentTok{\#summary(sls)}
\FunctionTok{screenreg}\NormalTok{(}\FunctionTok{list}\NormalTok{(}\AttributeTok{OLS =}\NormalTok{ model, }\AttributeTok{two\_SLS =}\NormalTok{ sls), }\AttributeTok{digits =} \DecValTok{4}\NormalTok{)}
\end{Highlighting}
\end{Shaded}

\begin{verbatim}
## 
## =======================================
##              OLS           two_SLS     
## ---------------------------------------
## (Intercept)    0.3962 *     -0.0647    
##               (0.1735)      (0.3762)   
## educ           0.1242 ***              
##               (0.0095)                 
## exp            0.0339 ***    0.0376 ***
##               (0.0050)      (0.0062)   
## male           0.2934 ***    0.2905 ***
##               (0.0458)      (0.0508)   
## ethblack      -0.1957 **    -0.1575    
##               (0.0713)      (0.0834)   
## ethhisp       -0.0974       -0.0695    
##               (0.1003)      (0.1129)   
## educ_fitted                  0.1530 ***
##                             (0.0228)   
## ---------------------------------------
## R^2            0.3539        0.2065    
## Adj. R^2       0.3476        0.1988    
## Num. obs.    520           520         
## =======================================
## *** p < 0.001; ** p < 0.01; * p < 0.05
\end{verbatim}

\begin{Shaded}
\begin{Highlighting}[]
\CommentTok{\#Nedenstående er 2SLS lavet i R}
\CommentTok{\#sls\_r = ivreg(learnings \textasciitilde{} educ + exp + male + ethblack + ethhisp | meduc + feduc + siblings + exp + male + ethblack + ethhisp, data = data)}
\CommentTok{\#summary(sls\_r)}
\end{Highlighting}
\end{Shaded}

\hypertarget{opgave-6---udfuxf8r-overidentifikationstestet.-hvad-konkluderer-du}{%
\subsection{Opgave 6 - Udfør overidentifikationstestet. Hvad konkluderer
du?}\label{opgave-6---udfuxf8r-overidentifikationstestet.-hvad-konkluderer-du}}

\hypertarget{opgave-7---udfuxf8r-hele-analysen-igen-hvor-du-kun-bruger-meduc-og-feduc-som-instrumenter.-uxe6ndrer-det-puxe5-dine-konklusioner}{%
\subsection{Opgave 7 - Udfør hele analysen igen hvor du kun bruger meduc
og feduc som instrumenter. Ændrer det på dine
konklusioner?}\label{opgave-7---udfuxf8r-hele-analysen-igen-hvor-du-kun-bruger-meduc-og-feduc-som-instrumenter.-uxe6ndrer-det-puxe5-dine-konklusioner}}

\end{document}
